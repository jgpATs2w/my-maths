\documentclass[10pt]{article}
\usepackage[utf8]{inputenc}
\usepackage[T1]{fontenc}
\usepackage{amsmath}
\usepackage{amsfonts}
\usepackage{amssymb}
\usepackage[version=4]{mhchem}
\usepackage{stmaryrd}

\begin{document}
Le sorprende que yo esté trabajando simultáneamente en literatura y matemáticas. Muchas personas que no han tenido nunca la oportunidad de aprender que son las matemáticas, las confunden con la aritmética y la consideran una ciencia árida y fría. El hecho es que es la ciencia que más imaginación necesita. Uno de los más grandes matemáticos de nuestro siglo dice muy acertadamente que es imposible ser matemático sin ser un poeta de espíritu. A mí me parece que el poeta debe ser capaz de ver lo que los demás no ven, debe ver más profundamente que otras personas. $Y$ el matemático debe hacer lo mismo.

\section*{ANÁLISIS MATEMÁTICO}
MARÍA MOLERO APARICIO

ADELA SALVADOR ALCAIDE

TRINIDAD MENARGUEZ PALANCA

LUIS GARMENDIA SALVADOR

\section*{CONTENIDO}
CONTENIDO ..... I
PRÓLOGO ..... $X I$
VARIABLE COMPLEJA ..... 1
HISTORIA DE LA VARIABLE COMPLEJA ..... 2
Los números complejos ..... 2
Funciones de variable compleja ..... 5
La función logaritmo ..... 6
Integración ..... 8
Cauchy y la variable compleja ..... 9
Riemann y la variable compleja ..... 12
Weierstrass y la variable compleja ..... 13
CAPÍTULO 1. Los números complejos ..... 17
1.1. EL CUERPO DE LOS NÚMEROS COMPLEJOS ..... 18
1.1.1. Números complejos en forma binómica ..... 19
1.1.2. Operaciones en forma binómica ..... 20
1.1.3. Propiedades algebraicas ..... 21
Ejemplos resueltos ..... 23
Ejercicios ..... 23
1.2. REPRESENTACIÓN GEOMÉTRICA. DIAGRAMA DE ARGAND ..... 25
Ejemplos resueltos ..... 27
Ejercicios ..... 29
1.3. FORMA POLAR ..... 30
1.3.1. Módulo ..... 30
1.3.2. Argumento ..... 31
1.3.3. Propiedades del módulo, del conjugado y del argumento de un número complejo ..... 32
1.3.4. Forma polar ..... 33
Ejemplos resueltos ..... 34
Ejercicios ..... 35
1.4. FORMA EXPONENCIAL DE UN NÚMERO COMPLEJO ..... 37
1.4.1. Operaciones entre números complejos en forma exponencial ..... 37
1.4.2. Fórmula de Moivre ..... 40
Ejemplos resueltos ..... 40
Ejercicios ..... 41
1.5. TOPOLOGÍA DEL PLANO COMPLEJO ..... 43
Ejemplos resueltos ..... 47
Ejercicios ..... 47
1.6. LA ESFERA DE RIEMANN. PROYECCIÓN ESTEREOGRÁFICA. ..... 48
Ejercicios ..... 51
1.7. EJERCICIOS ..... 52
CAPÍTULO 2. Funciones complejas ..... 57
2.1. DEFINICIÓN. FUNCIONES ELEMENTALES ..... 59
2.1.1. Definición de función compleja ..... 59
2.1.2. Funciones Elementales ..... 60
2.1.2.1. Polinomios ..... 60
2.1.2.2. Funciones racionales ..... 61
2.1.2.3. Función exponencial ..... 61
2.1.2.4. Funciones trigonométricas ..... 63
2.1.2.5. Funciones hiperbólicas ..... 65
2.1.2.6. Función logaritmo ..... 66
2.1.2.7. Funciones definidas como potencias ..... 68
Ejemplos resueltos ..... 70
Ejercicios ..... 74
2.2. LÍMITES DE FUNCIONES. CONTINUIDAD. ..... 76
2.2.1. Límites de funciones ..... 76
2.2.2. Límites en el infinito. Límites infinitos ..... 76
2.2.3. Continuidad. ..... 77
Ejemplos resueltos ..... 79
Ejercicios ..... 81
2.3. DERIVADA COMPLEJA ..... 82
2.3.1. Definición de derivada ..... 82
2.3.2. Propiedades ..... 85
2.3.3. Condiciones de Cauchy Riemann. ..... 86
2.3.4. Estudio de la derivada de distintas funciones ..... 89
Ejemplos resueltos ..... 91
Ejercicios ..... 93
2.4. FUNCIONES HOLOMORFAS ..... 94
2.4.1. Funciones holomorfas. Definiciones ..... 95
2.4.2. Estudio de la holomorfía de las distintas funciones ..... 95
2.4.3. Propiedades de las funciones holomorfas ..... 96
Ejemplos resueltos ..... 97
Ejercicios ..... 98
2.5. FUNCIONES ARMÓNICAS ..... 99
2.5.1. Funciones armónicas. Definición ..... 99
2.5.2. Propiedades de las funciones armónicas. ..... 101
Ejemplos resueltos ..... 102
Ejercicios ..... 103
2.6. EJERCICIOS ..... 104
CAPÍTULO 3. Series complejas ..... 111
3.1. SUCESIONES Y SERIES DE NÚMEROS COMPLEJOS ..... 113
Ejemplos resueltos ..... 117
Ejercicios ..... 119
3.2. SUCESIONES Y SERIES DE FUNCIONES COMPLEJAS ..... 120
3.2.1. Sucesiones de funciones complejas ..... 120
3.2.2. Series de funciones complejas. Definición y convergencia ..... 122
3.2.3. Series de funciones complejas. Continuidad y derivabilidad ..... 125
Ejemplos resueltos ..... 127
Ejercicios ..... 128
3.3. SERIES DE POTENCIAS ..... 129
3.3.1. Definición. Convergencia de una serie de potencias ..... 129
Ejemplos resueltos ..... 135
3.3.2. Funciones definidas por series de potencias ..... 136
Ejemplos resueltos ..... 141
Ejercicios ..... 144
3.4. FUNCIONES ANALÍTICAS ..... 145
3.4.1. Definición y propiedades ..... 145
3.4.2. Desarrollos en serie de funciones ..... 147
3.4.3. Prolongación analítica ..... 148
Ejemplos resueltos ..... 152
Ejercicios ..... 153
3.5. SERIES DE LAURENT ..... 154
3.5.1. Series de Laurent. Definición y convergencia ..... 154
3.5.2. Representación de funciones en series de Laurent ..... 158
Ejercicios ..... 163
3.6. EJERCICIOS ..... 163
CAPÍTULO 4. Integración en el plano complejo ..... 169
4.1. CURVAS EN EL CAMPO COMPLEJO. ..... 170
Ejemplos resueltos ..... 175
Ejercicios ..... 178
4.2. INTEGRACIÓN SOBRE CAMINOS. ..... 179
4.2.1. Integral de una función sobre un camino ..... 180
4.2.2. Relación de la integral compleja con la integral curvilínea real ..... 181
4.2.3. Propiedades elementales ..... 182
Ejemplos resueltos ..... 184
Ejercicios ..... 188
4.3. ÍNDICE DE UN PUNTO RESPECTO DE UNA CURVA. ..... 190
4.3.1. Definición de índice ..... 190
4.3.2. Índice y homotopía ..... 192
4.3.3. Índice y conexión ..... 193
Ejemplos resueltos ..... 193
Ejercicios ..... 195
4.4. TEOREMA DE CAUCHY. ..... 196
4.4.1. Primitivas ..... 196
4.4.2. Distintos enunciados del teorema de Cauchy. ..... 199
Versión primera del Teorema de Cauchy ..... 200
Lema de Goursat ..... 201
Teorema de Cauchy para un disco ..... 205
Teorema de Cauchy para caminos homótopos ..... 206
Teorema de Cauchy en dominios simplemente conexos ..... 207
Teorema de Cauchy-Goursat ..... 208
Ejemplos resueltos ..... 208
Ejercicios ..... 209
4.5. INTERPRETACIÓN FÍSICA Y GEOMÉTRICA DE LA INTEGRAL COMPLEJA ..... 211
4.5.1. Trabajo y flujo ..... 211
4.5.2. Teorema de la divergencia ..... 213


\end{document}